% ************************** Thesis Abstract *****************************
{
\cleardoublepage
\setsinglecolumn
\chapter*{\centering \LARGE Kurzzusammenfassung}
\thispagestyle{empty}
Alternde Menschen zeigen einen stetigen R\"uckgang der adaptiven Immunfunktion, mit wichtigen Auswirkungen auf Gesundheit und Lebensdauer. Systemische Ver\"anderungen, die in der Struktur und Diversit\"at des Antik\"orperrepertoires mit dem Alter beobachtet werden, spielen eine wichtige Rolle in diesem immunoseneszenten Ph\"anotypen; die relativ lange Lebensdauer der meisten Wirbeltier-Modellorganismen macht eine gr\"undliche Untersuchung des alternden Immunrepertoires jedoch schwierig. Als nat\"urlich kurzlebiges Wirbeltier bietet der T\"urkise Prachtgrundk\"arpfling (\textit{Nothobranchius furzeri}) eine aufregende neue Gelegenheit, die Alterung des adaptiven Immunsystems im Allgemeinen und des Antik\"orperrepertoires im Besonderen zu untersuchen.

In dieser Arbeit habe ich eine Kombination aus bestehenden genomischen Assemblierungen und neu generierten Sequenzierungsdaten verwendet, um den Immunoglobulin-H-Ketten-Locus des T\"urkisen Prachtgrundk\"arpflings zusammenzusetzen, zu charakterisieren und mit Loci eng verwandter Arten zu vergleichen. Dies zeigt eine dynamische Locus-Entwicklung mit sich wiederholenden Duplizierungen und Verlusten des spezialisierten mukosalen Isotypen \textit{IGHZ}. Diese Ergebnisse unterst\"utzen eine hohe Evolutionsrate der Immunoglobulin-H-Ketten-Loci in den Teleostei und bilden eine solide Grundlage f\"ur die Forschung der vergleichenden evolution\"aren Immunologie in Zahnk\"arpflingen.

Nachdem ich die Immunoglobulin-H-Ketten-Locus-Sequenz in \textit{N. furzeri} charakterisiert hatte, nutzte ich sie, um Immunoglobulin-Sequenzierung gezielt f\"ur diese Spezies zu etablieren, welche eine quantitative Analyse des Antik\"orperrepertoires erm\"oglicht. Die Anwendung dieses Protokolls auf Ganzk\"orper-Fischproben ergab komplexe und individualisierte Antik\"orperrepertoires, die in der intraindividuellen Diversit\"at mit dem Alter rasch abnehmen und in der interindividuellen Variabilit\"at zunehmen. Dies zeigt, dass die Antik\"orperrepertoires von T\"urkisen Prachtgrundk\"arpflingen, entsprechend ihrer kurzen Lebensdauer schnelle Alterungsprozesse aufweisen. Dieser altersbedingte Diversit\"atsverlust war besonders stark in Darmproben - ein Ph\"anomen, das mit der konstanten ausgepr\"agten Antigenexposition an Schleimhautoberfl\"achen zusammenh\"angen kann und bisher nicht in einem Wirbeltier-Modell untersucht wurde. Zusammenfassend etablieren diese Ergebnisse den T\"urkisen Prachtgrundk\"arpfling als neuartiges Modell f\"ur die Erforschung von Immunoseneszenz in Wirbeltieren und bilden die Grundlage f\"ur zuk\"unftige Analysen von und Interventionsstudien an adaptiver Immunalterung.
}

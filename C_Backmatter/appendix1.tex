%!TEX root = ../thesis.tex
% ******************************* Thesis Appendix A ****************************
\chapter{Principles of repertoire sequencing analysis}

% Appendix for now; may move to intro or methods later

The analysis of repertoire sequencing data poses a number of challenges in addition to those inherent in any high-throughput-sequencing-based project. 

\subsection{Preprocessing repertoire sequencing data}

As with any sequencing protocol, errors and biases accumulate at every stage of Ig-Seq library preparation, which if not corrected for can seriously affect the final result of any analysis. In the case of repertoire sequencing, however, these errors are both

The highly quantitative nature of Ig-Seq analysis makes tackling these errors particularly important, while the numerous real deviations from the germline sequence -- arising through junctional diversification and somatic hypermutation -- make it particularly difficult to distinguish real from erroneous sequence variation using bioinformatic methods alone. 

Errors in the sequencing dataset can arise from both the PCR reactions required to prepare and amplify the library and the sequencing process itself, as well as 


...

These deviations can be divided into two main categories: sequence errors and biases in relative sequence abundance.
The deviations that accumulate during 

Sequence errors can accumulate from various sources, the most important of which are base incorporation errors during reverse-transcription and PCR and base-calling errors during sequencing. Abundance bias, meanwhile, can occur due to differential template-binding affinity of multiplexed primers, differences in polymerase binding affinity and extension speed during reverse-transcription and PCR, differential retention of different sequences during bead-based cleanup, and differential binding of different sequences to the Illumina sequencing flowcell. % Read up on sequencing errors/bias and get citations for everything
In both cases, deviations that occur before or during PCR amplification have the greatest effect, as early errors and small biases can propagate exponentially through later cycles, potentially dominating the resulting library.
 

Many of these issues can be mitigated, though not eliminated, through experimental methods. 

The use of a RACE-based template-switching protocol avoids the need for multiplexed variable-region primers and the substantial PCR and sequencing biases introduced thereby. Overlapping paired-end sequencing protocols enable the quality of the terminal parts of each read to be improved through merging, substantially improving the quality of the overlapping region. 

% Explain issues with RACE-based protocol

% Explain importance of UMI-based approach



The analysis of immunoglobulin sequencing datasets can be broadly split into three main stages, each of which makes use of a different suite of tools and techniques. In the first step, 

\subsection{Error and bias correction with UMIs: pRESTO}

The preprocessing stage of Ig-Seq analysis is well-illustrated by the pRESTO pipeline, one of the earliest comprehensive tools for repertoire sequencing analysis.

\begin{enumerate}
\item Quality filtering: Mask or trim low-quality sequences, and/or discard reads with low overall quality.
\item Barcode annotation: Identify and annotate each read with its sample index and UMI sequences, if available. % Identification modes, optional trimming
\item Match read pairs: Copy/merge index and barcode annotations across read pairs
\item Align MIGs: Group reads with matching barcodes and construct a multiple sequencing alignment within each group; quantify nucleotide diversity
\item Generate consensus reads: Infer a single consensus sequence for each MIG, representing a single input molecule; estimate error rates
\item Merge read pairs: Align and merge paired consensus reads into a single, higher-quality sequence % Why do this after consensus read inference?
\item Collapse matching sequences: Remove consensus reads with identical sequence from the read set; optionally annotate each remaining sequence with the number of distinct UMIs labelling that sequence
\item Partition read set: Separate/filter reads based on sample index or other annotations, or generate random subsets of reads.
\end{enumerate}

% Figure of pRESTO's paired-end UMI-based workflow
% Add pRESTO paper to bibliography

Many of these stages can be performed in a variety of ways depending on user requirements; see \dots for more details on the specific algorithms used.

The output of pRESTO is therefore a collection of filtered consensus reads, annotated according to their abundance in the original read set and demultiplexed by sample of origin. These consensus reads can then be assigned V/D/J and constant-region identities (section \dots) and used in downstream analysis of repertoire composition and structure (section \dots) as well as a variety of other applications.

\dots

The generation of consensus reads from UMI groups in the pRESTO pipeline effectively eliminates a great deal of errors and biases introduced during sample preparation and sequencing, especially when the sequencing depth is sufficiently high. However,  there are some categories of error that are not adequately handled by UMI merging. These less tractable errors can be divided into three broad categories, each of which poses particular problems for error correction: early errors, recurrent errors, and barcode errors.
\begin{itemize}
\item \textbf{Early errors:} Sequence errors which accumulate in cDNA before the addition of UMIs (i.e. during reverse-transcription or template switching) will naturally propagate to all members of a MIG during library preparation, and will therefore dominate the consensus read following MIG condensation. Similarly, errors which occur during sufficiently early PCR cycles will also be strongly amplified during later cycles, producing either an MIG dominated by an erroneous sequence, or an ambiguous UMI without a dominant sequence. Ambiguous MIGs will typically be discarded during error correction; those with a dominant erroneous sequence will be collapsed into an incorrect consensus read \citep{shugay2014migec}.
\item \textbf{Recurrent errors:} It is well-established %citations needed
that particular sequence contexts are especially prone to producing particular errors during reverse-transcription, PCR, or other polymerase-based reactions. These 

Sequence errors which accumulate in cDNA prior to the attachment of UMIs -- i.e. during reverse transcription or template switching -- 
\item Dominant errors: Errors which occur sufficiently early in the library preparation process
\end{itemize}
These less tractable errors can be divided into n broad categories: those that occur before UMI attachment (i.e. during reverse-transcription or template switching); those that occur early in the amplification process (and therefore come to dominate the MIG); those that occur regularly in a given sequence context (``hotspot" errors); and, most intractably, those that affect the barcode sequences themselves. Each of these special classes of error poses a particular challenge in Ig-Seq pre-processing, and is dealt with (or not) differently by different tools. % Separate and expand this paragraph?

\subsection{Tackling early and recurrent errors: MIGEC}

Like pRESTO, MIGEC \citep{shugay2014migec} is a tool for pre-processing UMI-tagged repertoire sequencing reads. As with pRESTO, MIGEC extracts UMI information for each read pair in the library, groups reads by UMI sequence and generates consensus reads for each MIG. 

\subsection{Tackling barcode errors: IgReC}
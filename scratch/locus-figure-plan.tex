IgH loci introductory figures

Figure 1: Antibodies are the antigen-receptor proteins of the vertebrate humoral immune system
 - Physical and schematic structures of an idealised antibody
 - Definition and illustration of FR, CDR, isotype, idiotype, constant & variable region, effector domain, Fab and Fc (glossary?)
 - Illustration of antigen-binding loops (collier de perles?)
 - Roles of antibody in the immune system (examples from mammals and fish, e.g. opsonisation, aggregation, complement, etc)
 
Figure 2: Naïve antibody diversity arises through VDJ recombination, junctional diversity and B-cell selection
 - Schematic illustration of VDJ recombination on a translocon locus (V,D,J -> contiguous region + junctions)
 - Illustration of junctional diversification (palindromic and non-palindromic insertions, deletions, enzymes involved)
 - Initial B-cell selection in mammals and fish?
 
% Structure of IgH and IgL loci therefore underlies potential diversity of antibody proteins in an organism - determines possible V/D/J sequences and number of possible combinations
% Examples of V segments known to be associated with particular things?

Figure 3: IgH loci are highly variable in size and structure
 - Simple to-scale schematics of locus sizes in different species (mammals, fishes, others?)
 - Mouse/human locus schematics
 - Several example fish (esp. my reference species)
 
Figure 4: V, D, \& J regions in (mammals and) teleosts 
 
Figure 5: Antibody constant regions in (mammals and) teleosts
 - Different isotypes differ in length, number of exons, hinge region, and effector functions
 - C vs TM exons
 - Each C exon typically represents a complete immunoglobulin fold domain (except hinges?)
 - IgM, D and Z/T in teleosts
 - Crucial features (e.g. inter- and intra-chain cysteine pairs)

% Something about recombination signal sequences

\section{Results and discussion}
\sec{sec:locus_results}

\section{The 

In order to characterise the immunoglobulin heavy chain (\textit{IgH}) locus of the turquoise killifish, gene segments were obtained from the 
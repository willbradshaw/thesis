\section{The vertebrate adaptive immune system} % Humoral/B-cell/antibody immune system?

All organisms exist in a condition of intense competition for resources, with predators, peers, and parasites all competing, in some way, for the nutrients and energy consumed and used by an individual. Among the most insidious of these competitors are parasites who attempt to colonise an organism's own body, consuming its stores of nutrients and energy and turning its internal mechanisms to their own advantage. When an organism falls prey to one of these parasites and manifests the symptoms of its exploitation, we call it disease. When the organism utilises adaptations to prevent this exploitation, through excluding or killing the parasites, we call it immunity.

Given the extreme selective pressure to protect their fitness from parasitic exploitation, it is perhaps unsurprising that so many different organisms have evolved immune systems of great complexity and effectiveness. Nevertheless, the intricacy of the vertebrate immune system has proven one of the most enduringly fascinating aspects of vertebrate biology, comparable to vertebrate neural systems in its complexity. Indeed, the vertebrate immune system shows many parallels with the nervous system, being the only other system capable of complex information processing and memory. This complexity, which is fundamental to the effectiveness of the vertebrate immune system, rests on the interplay between the two traditional wings of vertebrate immunology: the innate immune system, and the adaptive.

Innate immunity refers to a large collection of mechanisms designed to exclude, sequester, or kill invading pathogens (disease-causing organisms) in a rapid and nonspecific manner. Many innate systems combat pathogens in ways that are either physically difficult to circumvent (such as external barriers, or engulfment by phagocytic cells) or which target aspects of pathogen biology that are difficult to alter without catastrophic loss of function (such as this wonderful example %example
). As such, the great majority of possible pathogenic threats are dealt with rapidly and effectively by the innate immune system, either by keeping parasites from accessing vulnerable parts of the organism or by rapidly and nonspecifically eliminating them once there.

Despite its speed, power and impressive generality, the innate immune system suffers from severe limitations. The first is that it is helpless in the face of evolutionarily novel threats to which its existing defences do not apply. The second, perhaps more fundamental, problem is that many common pathogens are capable of evolving at speeds vastly exceeding that of vertebrates. For example, many bacteria have generation times of much less than an hour, compared to months or years for most vertebrates, while also exhibiting much higher per-cell-division rates of mutation. And even this rapid rate of evolution is far exceeded by the highly volatile genomes of many viruses.

This capacity for parasitic organisms to out-evolve their hosts represents a serious problem for vertebrates, who cannot hope to effectively respond to these threats through the generation of new and improved innate immune mechanisms via selection. Instead, what is needed is a mechanisms by which vertebrates can dynamically learn to respond to novel immune threats within the lifespan of a single organism. That mechanism is the adaptive immune system.

% Need to rapidly focus onto B-cell immunity in particular, I know nothing of T-cells

The mechanisms used to generate this sequence diversity in the B-cell population are themselves diverse, and have been discovered progressively over the past decades. The most fundamental, and well-known, such mechanism is so-called V(D)J recombination, first discovered quite some time ago. %Find out more about history of this

In V(D)J recombination, a number of

A canonical immunoglobulin heavy chain (IgH) gene locus consists of clusters of variable (V), diversity (D) and joining (J) regions in series, followed by some number of larger constant-region exons. During B-cell maturation, a single V, D and J region are selected and the intervening DNA regions are excised to produce a single, contiguous VDJ sequence. As part of this process, nontemplated nucleotides are inserted and deleted at the V/D and D/J junctions, a process known as junctional diversification. Following transcription, the sequence between the VDJ sequence and the first constant-region exon is removed by splicing to produce a mature IgH transcript, with its characteristig VDJC sequence structure.

The manner of constant region selection in B-cells differs significantly between mammals and teleosts. In the former, a number of distinct constant regions are present in series on the chromo

\newpage 
\section{Structure and diversification of the antibody heavy chain}

% Pretend light chains and TCRs don't exist for now; you can generalise later as needed

The great majority of antibodies produced by the gnathostome adaptive immune system share a canonical tetrameric structure: two heavy chains and two light chains, arranged into a roughly Y-shaped configuration. This structure comprises three important functional domains: two antigen-binding domains, formed by the N-terminal portions of the heavy and light chains, and one effector domain formed by the C-termini of the heavy chains. As a result of these distinct functionalities, the two ends of an antibody heavy-chain protein have very different properties when considered as a population. The N-terminal variable domain is highly variable in amino-acid sequence, with unrelated\footnote{i.e. not derived from the same plasma-cell clone} B-cells typically displaying different and often unique sequences. The C-terminal constant region, meanwhile, shows very limited sequence diversity, with all B-cells in the body producing one of a small number of distinct effector classes. This combination of a highly-diverse antigen-binding domain and a limited range of distinct effector domains enables antibodies to simultaneously recognise and contact a vast array of potential antigens, while simultaneously interacting with the rest of the immune system in a predictable manner.

The constant-region sequence of an antibody heavy chain (or its mRNA) is known as the class or isotype of that antibody, while its variable-region sequence is known as its idiotype. % Definitions box?

The variable region of an antibody can be further subdivided into three complementarity-determining-regions (CDR1-3), which form part of the antigen-binding site and directly contact the antigen, and four framework regions (FR1-4), which do not. Sequence variability in the variable region is concentrated in the CDRs, with CDR3 showing by far the greatest variability for a variety of reasons discussed below. % Figure for this? E.g. align Vs and mark regions of non-conservation
This sequence variability is generated by several highly-specialised genome-editing mechanisms, which together produce a distinct idiotypic sequence on the chromosome of each developing B-cell. Broadly speaking, these diversification mechanisms can be divided into three categories, each of which has a distinct effect on repertoire sequence diversity; these categories are V(D)J recombination, junctional diversification, and somatic hypermutation.

\subsection{Mechanisms of heavy-chain diversification I: V(D)J recombination}


% FIGURES AND TABLES:
% - Schematic of native locus state -> VDJR -> recombined locus
% - RAG structure and DNA Binding
% - Schematic of RAG action and DNA excision
% - Schematic of RSS structure
% - Sequence logos of human/mouse/other RSSs
% - Schematic of V/D/J coverage on protein chain
% - Length distributions of human/mouse/other V/D/J segments

The structure of the immunoglobulin heavy chain (\textit{IGH}) locus differs markedly between its native germline configuration and that adopted by a mature B-cell. In the native state, most \textit{IGH} loci comprise large numbers of isolated gene segments separated by non-coding DNA. These segments can be divided into three classes, variable (V), diversity (D) and joining (J), with distinct sequence properties and length distributions (see Box~\dots). % FIGURE: VDJ length distributions in humans etc; schematic of V/D/J positioning in peptide chain
During B-cell development, site-specific recombination reactions result in the rearrangement of individual V, D and J segments into a continuous VDJ sequence, with the intervening genomic regions permanently excised and degraded. % Citation about what happens to excised sequences
This irreversible genomic maturation process is known as V(D)J recombination.

VDJ recombination is carried out by the RAG endonuclease complex, an enzyme formed by the association of \textbf{r}ecombination-\textbf{a}ctivating \textbf{g}enes 1 \& 2 \citep{jung2006vdjr}. % more detailed citation, roles of RAG1 vs RAG2
This complex, which is expressed specifically in developing lymphocytes, % citation needed
introduces recognises specialised recombination sequences (RSSs) at the ends of IGH gene segments and introduces targeted double-strand breaks between two segments and their respective RSSs \citep{jung2006vdjr}; these DSBs are then repaired by non-homologous end joining, resulting in a continuous coding sequence spanning both segments. The excised DNA ... % Citation for this

VDJ recombination in the \textit{IGH} locus is highly structured, and occurs in a specific order. First, a D and a J segment are selected and recombined to produce a DJ sequence. Then, a V region is selected and recombined with the DJ to produce a continuous VDJ sequence constituting the variable region of the heavy chain. The complete protein sequence is produced during later transcriptional splicing, which joins this variable-region sequence to downstream constant-region exons to produce a mature \textit{IGH} mRNA. This strict ordering of V/D/J segments, which obtains in the vast majority of recombined sequences observed, is produced through the combination of a variety of regulatory mechanisms. The most basic of these is the structure of the RSSs, which comprise a conserved heptamer and nonamer sequence separated by a relatively unconserved spacer region of either 12 or 23bp \citep{jung2006vdjr}, corresponding to either one or two turns of the DNA helix. V and J segments in the IGH locus are flanked by RSSs with 23bp spacer regions, while those flanking D-regions have 12bp spacers %citation needed
As the RAG recombinase specifically recognises pairs of RSSs with dissimilar spacer lengths (a restriction known as the 12/23 or one-turn/two-turn rule), direct V-to-J recombination events are excluded \citep{jung2006vdjr}. % Better citation if possible.

In addition to the restrictions imposed by the 12/23 rule, additional limitations on VDJ recombination are imposed by the requirement that RAG binding be preceded by transcription from the to-be-recombined segments, apparently in order to open the chromatin structure in that region \citep{jung2006vdjr}. Both V and D segments, but not J segments, are preceded by upstream promoter regions, which are involved in transcriptional initiation during specific stages of B-cell development...

Complete VDJ recombination places the V-region promoter in close proximity to a highly-conserved enhancer element (known as iE$\mu$) lying between the last J segment and the first constant-region exon \citep{jung2006vdjr}; this enhancer is important for strong expression of the mature IGH mRNA from the pre-B-cell stage onwards. 

\begin{itemize}
\item \textbf{Variable (V) segments} are the longest class of \textit{IGH} gene segment, with functional segments ranging from \dots to \dots nucleotides in humans and mice. % Figure and citation for this; add amino-acid lengths
This includes the coding sequence for the bulk of the variable region of the heavy chain, including all of FR1, CDR1, FR2, CDR2 and FR3, as well as the 5'/N-terminal part of CDR3 \citep{jung2006vdjr}. Each V segment also includes an N-terminal signal peptide sequence for antibody trafficking % citation needed
, and is preceded by its own 5'-UTR and promoter region. % citation for V promoters?
\item \textbf{Diversity (D) segments} are the shortest class of \textit{IGH} gene segment, ranging from \dots to \dots nucleotides in humans. %!
They form the middle part of the heavy chain CDR3. Like V-segments, D-segments are flanked by upstream promoter regions, which are involved in initiating the transcription required for DJ recombination. \citep{jung2006vdjr}
\item{Joining (J) segments} are of intermediate length, \dots % length data
. They form the 3'/C-terminal part of heavy chain CDR3 and the 5'/N-terminal part of FR4. Each J-segment is succeeded on the chromosome by a splice donor site, which is used to splice the recombined variable region to the constant region following transcription of IGH mRNA.
\end{itemize}

In its native germline state, the antibody heavy chain locus occupies a highly unusual configuration



The immunoglobulin heavy chain (\textit{IGH}) locus has a highly distinctive structure, whose broad outlines are shared only by other adaptive-immune antigen receptors...

The human genome contains seven loci that follow these general patterns of diversification: the \textit{IGH} locus, two light chain loci, and four T-cell receptor loci \citep{jung2006vdjr}.

% First explain structure of antibody protein chain
% -> mRNA
% -> unrecombined locus



An antibody protein chain can be divided into two regions with distinctive roles in the immune system: an N-terminal antigen-binding domain

. The C-terminal part of an antibody, known as the constant region, 

The N-terminal portion, by contrast, 

While the first two complementarity-determining regions of the heavy chain are contained entirely within the V segment, the heavy-chain CDR3 is formed by the combination of the V, D, and J segments selected for rearrangement \citep{jung2006vdjr}. As a result, the bulk of sequence diversity in the heavy chain falls within the CDR3, which is responsible for the greater part of antigen-binding variability among antibodies. % Citation for this
Nevertheless, the combinatorial sequence diversity produced by rearrangements of V/D/J segments alone is necessarily limited, with a maximum of $a \times b \times c$ possible sequences for a locus containing $a$ V, $b$ D and $c$ J sequences; far below the diversity of possible antigens. % citation needed 
Fortunately, the potential sequence diversity of the na\:{i}ve repertoire is powerfully augmented by a second diversification mechanism, taking place alongside VDJ recombination during B-cell development: junctional diversification.

\subsection{Mechanisms of heavy-chain diversification II: junctional diversification}

...

The net number of inserted nucleotides is essentially random, but the reading frame of the IGH transcript is fixed by the positions of the ATG initiation codon in the V-segment and the J-segment/constant-region splice junction. As a result, junctional diversification results in a large number of frameshift mutations, in addition to STOP codons that prematurely truncate the protein sequence. % Are these STOP mutations also counted as non-productive, or just frameshifts
Such loci, which are unable to produce a functional heavy chain protein, are termed non-productive, while those without such mutations are termed productive ; at a first approximation, roughly one-third of VDJ recombination events are expected to produce a productive sequence \citep{jung2006vdjr}. % Derivation and graphical representation of this; probability of two discrete random variables being congruent modulo three?

While a given recombination event only has about a one-third chance of producing a productive rearrangement, B-cells, like other somatic cells in vertebrates, are diploid. As a result, while a given recombination event only has about a one-third chance of producing a productive rearrangement, a B-cell which undergoes a non-productive rearrangement on one chromosome can make a second attempt on the other. Given the one-third approximation given above, this means that about 55\% of B-cells will achieve a productive rearrangement on one or the other chromosome; the other 45\%, unable to produce a functional heavy chain from either chromosome, die by apoptosis during B-cell development \citep{jung2006vdjr}. % Adapt figure 3 from \citep{jung2006vdjr}
This loss of almost half of all developing B-cells, a substantial cost, demonstrates the profound selective value of the additional antigen-binding diversity provided by junctional diversification.

Of those B-cells undergoing a productive heavy-chain rearrangement (and therefore surviving this process), roughly three-fifths are expected to bear one rearranged and one unrearraranged locus, with the remaining two-fifths bearing two rearranged loci, one of which in unproductive. This 60/40 ratio is roughly borne out by empirical data 


\subsection{Structure of mammalian \textit{IGH} loci} % And teleosts/others as available

% Human locus

The mouse \textit{IGH} locus also adopts a translocon structure, in this case with an enormous V-region, comprising 150 or more V-segments (depending on the strain) spanning 2.7Mb \citep{jung2006vdjr}. The 12-13 murine D segments occupy a region of approximately 50kb, while the four J segments cover about 2kb. This is followed by 200kb of constant region exons. 

% Mouse locus (depending on strain): 150+ V, 12-13 D, 4 J. (\citep{jung2006vdjr})
% Total length of murine locus: ~3Mb near telomeric end of chr12

\subsection{Antibody effector function and isotype diversity}



\newpage
\section{The African turquoise killifish as a model for vertebrate ageing}

% POSSIBLE FIGURES:
% - Nothobranchius genus range, photos of different notho males
% * Photo of male and female GRZ TK
% * Lifespan curves of male and female GRZ-AD
% - Diapause schematic
% - Photograph of ephemeral pools in Zimbabwe
% - Something showing correlation between aridity and lifespan in nothos
% - Schematic comparing GRZ lifespan to other common ageing models, along with important present/absent human-like systems
% - Phylogeny of TK within teleosts, with divergence times and human outgroup

% POSSIBLE TABLES:
% - List of notho species used in research with approx mean/max lifespans
% - List of ageing phenotypes observed/not observed in GRZ / wild-derived strains
% - Table of TK genome assemblies (JENA/STFD/CLGN) with metrics
% - Comparison of genome metrics (size, repetitiveness, etc) between TK and other models
% - Life history table for GRZ, MZM, other nothos, other fish models

The genus \textit{Nothobranchius} comprises a broad group of annual freshwater fishes distributed across equatorial and subequatorial Africa \citep{valdesalici2003lifespan}, with species diversity concentrated in the south-east of the continent \citep{genade2005annual}. Members of this genus share a suite of adaptations to life in ephemeral pools and rivers, most notably the production of desiccation-resistant embryos capable of surviving through the dry season in a diapause state \citep{genade2005annual}. Fish from this genus have been known for several decades to exhibit very rapid growth and short lifespans, consistent with their evolving under conditions of very high extrinsic mortality \citep{valdesalici2003lifespan}, with many species exhibiting a median lifespan of less than one year. Nevertheless, there is wide variation within the genus in body size, growth rate and lifespan, with species from less arid regions tending to show slower growth and longer median lifespans \citep{genade2005annual}.

Like other \textit{Nothobranchius} species, the turquoise killifish (\textit{Nothobranchius furzeri}) is a medium-sized annual fish first isolated from ephemeral freshwater pools -- in this case, from a relatively arid region of southeastern Zimbabwe \citep{jubb1971new,genade2005annual}. Even by the standards of the \textit{Nothobranchius} genus, \textit{N. furzeri} exhibits extremely rapid growth, maturation, and ageing, with the most widely-used laboratory strain (GRZ) exhibiting a median lifespan of just 9-14 weeks \citep{valdesalici2003lifespan,genade2005annual,terzibasi2008strains,kirschner2012map,valenzano2015genome} % Update range as other papers get different results
-- the shortest lifespan of any captive vertebrate. Moreover, while all turquoise killifish strains are very short-lived by vertebrate standards, different strains of this species have been found to differ several-fold in their median and maximum lifespans \citep{terzibasi2008strains,kirschner2012map}, providing a rare opportunity for intraspecific comparative ageing studies \citep{terzibasi2008strains,terzibasi2009dr,hartmann2009telomeres} and mapping the genetic underpinnings of lifespan \citep{kirschner2012map}. This combination of extremely short-lived laboratory strains and a wide range of lifespan phenotypes within a single species makes the turquoise killifish an extremely promising model organism for ageing research, especially when combined with the presence of vertebrate-specific adaptations absent in short-lived invertebrate ageing models.


Despite its very short lifespan, \textit{N. furzeri} has been found to show a wide range of senescent phenotypes in even the shortest-lived strains, including lipofuscin deposition \citep{genade2005annual};  accumulation of senescence markers \citep{genade2005annual};  increased neurodegenaration \citep{valenzano2006resveratrol1,valenzano2006resveratrol2}; impaired learning and behavioural phenotypes  \citep{genade2005annual,valenzano2006resveratrol1}; and a high incidence of degenerative and neoplastic lesions \citep{dicicco2011histopathology}. These diverse phenotypes indicate that the short lifespan of the turquoise killifish is the result of an accelerated general ageing process, rather than the specific failure of a particular organ or system.
Moreover, established anti-ageing interventions such as resveratrol treatment \citep{valenzano2006resveratrol1}, reduction in ambient temperature \citep{valenzano2006temperature} and dietary restriction \citep{terzibasi2009dr} also extend lifespan in the turquoise killifish, indicating a strong analogy with the ageing phenotypes observed in canonical model systems.


% (though others, such as shortened telomeres \citep{hartmann2009telomeres} and gonadal fibrosis \citep{dicicco2011histopathology}, are only observed in longer-lived strains of the species) % Add more as you read them
% Other ageing phenotypes, including
% telomere shortening \citep{hartmann2009telomeres}, gonadal fibrosis \citep{dicicco2011histopathology}, % not observed in GRZ
% mtDNA depletion \citep{hartmann2011mitochondria}, and bioenergetic impairment \citep{hartmann2011mitochondria} % Not studied in GRZ
% have been observed in longer-lived \textit{N. furzeri} strains, but were either not observed (telomeres, gonads) or not studied (mitochondrial phenotypes) in the short-lived GRZ strain.
%.  

Due primarily to its potential as a model organism for ageing research, the turquoise killifish has also seen rapid development as a genetic model. The short-lived GRZ strain has been bred in captivity for fifty years and at least a hundred generations \citep{terzibasi2007review} and exibits a very high degree of homozygosity \citep{reichwald2009genome,valenzano2009map,kirschner2012map}, providing a uniform genetic background for experimental interventions. A variety of effective transgenesis and mutagenesis methods have been developed for this species, including \textit{tol2} transgenesis \citep{valenzano2011tol2,hartmann2012tol2,allard2013inducible} %TALENS?
and CRISPR \citep{harel2015crispr,harel2016crispr}; furthermore, a number of important genetic resources are now available, including linkage maps \citep{valenzano2009map,kirschner2012map,valenzano2015genome}, a transcript catalogue \citep{petzold2013transcriptome}, an miRNAome % citation needed
and a mitochondrial genome \citep{hartmann2011mitochondria}. Most importantly, a high-quality nuclear genome of the short-lived GRZ strain is now available
% with a recent study integrating evidence from multiple independent studies to produce a very high-quality assembly \cite{ray's genome paper, plus others}. 
\citep{reichwald2015genome,valenzano2015genome}. % Add more here
The progressive release of improved genome assemblies for the turquoise killifish has been particularly important for this project, and will be discussed in more detail in Section ??. %!
The genome itself is unusually large and repetitive by teleost standards, potentially contributing to these fishes' unusually short lifespan. % citation needed

% Needed: oxidation levels, telomere dysfunction, more senescence markers

% Ageing-related genes under positive selection (both genome papers)


%The killifish genome is unusually large by teleost standards, % TODO: figure for this? update figures and add citations as you read newer sources
%with an estimated size of 1.5 to 2 gigabases \citep{reichwald2009genome,reichwald2015genome}. This large size is primarily accounted for by the exceptionally high repeat content of the killifish genome, with at least 21\% and 24\% composed of tandem and other types of repeats, respectively \citep{reichwald2009genome}. GC content is also relatively high, at 44\%, though this average is affected somewhat by the presence of a distinct fraction of highly GC-rich tandem repeats \citep{reichwald2009genome}. Karyotyping \citep{reichwald2009genome} and sequencing analysis \citep{reichwald2015genome} both indicate a chromosome number of $2n = 38$, corresponding to 19 distinct linkage groups in the haploid genome. 

%They are strongly sexually dimorphic, with an XY sex-determination system \citep{reichwald2015genome,valenzano2009map}. %Photo

Phylogenetically, the genus \textit{Nothobranchius} falls within the Cyprinodontiformes, and the turquoise killifish is therefore closely related to a number of fish species that have been the subject of extensive research, including guppies, platyfish, medaka, sticklebacks, and pufferfish  \citep{terzibasi2007review} %TODO: cite an up-to-date phylogeny, add estimated divergency times, add a phylogeny figure.
; the genetic resources available for these species, in combination with the high degree of synteny shown across this group of teleosts \citep{terzibasi2007review} have been of great utility in a number of killifish projects, including this one. On the other hand, the most well-developed teleost model organism, the zebrafish, is relatively distantly-related, making the use in a killifish context of genetic and experimental resources developed for the zebrafish more challenging.


% TODO: Add a paragraph on the killifish immune system to segue into teleost immunity stuff

\section{Misc.}

\subsection{Ageing of the antibody repertoire}

% Notes from de Bourcey et al. (2017)

The vertebrate adaptive immune system has long been observed to undergo a severe decline with age in multiple species, with notable changes in humans including decreased lymphocyte proliferation \citep{debourcy2017ageing} and defects in antibody production \citep{debourcy2017ageing}. Changes in the human antibody repertoire observed with age include restricted

In studies of peripheral blood repertoires taken before and after influenza vaccination, older individuals have been observed to show reduced within-individual and increased between-individual repertoire diversity \citep{debourcy2017ageing}, suggesting that repertoires become increasingly (and divergently) specialised with age. Older repertoires also show less change in composition following vaccination, showing a reduced capacity to adapt to new information from the pathogenic environment \citep{debourcy2017ageing}. An oligoclonal phenotype, in which one or a few memory B-cell lineages occupy a disproportionate share of repertoire diversity, has been reported in a subset of older individuals in multiple studies \citep{debourcy2017ageing}; these expanded clones appear to be resistant to immunogenic interventions such as vaccination. 

Within similarly-sized lineages, the repertoires of older individuals have been found to show reduced per-nucleotide sequence diversity, indicating an impairment in secondary diversification through somatic hypermutation; this effect is especially pronounced in larger clones \citep{debourcy2017ageing}. % Fig 3D

A subset of older repertoires also showed a greater prevalence of sequences bearing premature stop codons, indicating... % Also reduced "radical" mutations

This loss in repertoire diversity with age is observed in both naive and antigen-experienced subsets of the repertoire \citep{debourcy2017ageing}, but is strongest in the former, indicating an increase in the relative prevalence of the memory compartment within the repertoire. This is consistent with a model of the aged immune system as impaired by the accumulation of stubborn immune memory.

An important feature of these studies is that CMV infection has been found to have an important influence of certain aspects of repertoire ageing, including...

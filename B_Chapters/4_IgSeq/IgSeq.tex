%!TEX root = ../thesis.tex
% TODO: Change this?

\chapter{Immunoglobulin sequencing in \textit{Nothobranchius furzeri}} 
\label{chap:igseq} 
\onehalfspacing
\pagebreak

\section{Introduction}
\label{sec:igseq_intro}

\dots

\section{Collecting killifish samples for immunoglobulin sequencing}
\label{sec:igseq_samples}

To obtain samples for establishing and validating an immunoglobulin-sequencing protocol in turquoise killifish, as well as to investigate changes in killifish repertoires with age, male GRZ-AD turquoise killifish from a single hatching cohort were raised under standard husbandry conditions (\Cref{sec:methods_husbandry}) and sacrificed by anaesthesia, followed by flash-freezing in liquid nitrogen and preservation at \degC{-80}. In total, thirty-two fish were sacrificed at four total time points (\Cref{tab:igseq-cohorts-summary,tab:igseq-cohorts-fish}): regular groups of ten fish each were sacrificed at roughly 5.5 weeks (shortly after reproductive maturation), 8 weeks (middle adulthood) and 10.5 weeks (close to median lifespan) post-hatching, while two surviving fish were sacrificed eight weeks later at roughly 18 weeks post-hatching.

\begin{table}
\caption{Summary of killifish used in \igseq validation and ageing experiment. All fish are GRZ-AD strain and male.}
\label{tab:igseq-cohorts-summary}
% latex table generated in R 3.5.2 by xtable 1.8-3 package
% Wed Jan 30 11:19:43 2019
\begin{tabular}{rrlllll}
  \toprule Group & \# Fish & Hatch date & Sacrifice date & Age (days) & Age (weeks) & Mean weight (g) \\ 
  \midrule 1 & 10 & 2016-05-09 & 2016-06-17 & 39 & 5.57 & 1.3 \\ 
  2 & 10 & 2016-05-09 & 2016-07-04 & 56 & 8 & 1.37 \\ 
  3 & 10 & 2016-05-09 & 2016-07-21 & 73 & 10.4 & 1.76 \\ 
  4 & 2 & 2016-05-09 & 2016-09-14 & 128 & 18.3 & 2.3 \\ 
   \bottomrule \end{tabular}

\end{table}

In order to obtain a representative sample of the whole-body antibody repertoires from these individuals (as opposed to the specific repertoire of a particular organ), the frozen fish were homogenised in liquid nitrogen with a mortar and pestle, and total RNA was isolated from the resulting powder with guanidinium thiocyanate-phenol-chloroform extraction (\Cref{sec:methods_molec_standard_qiazol}).

\section{\igseq in the turquoise killifish: principles and protocols}
\label{sec:igseq_protocol}

\subsection{Preparing the sequencing library}
\label{sec:igseq_protocol_library}

% TODO: Add general description of what IgSeq is and why you might do it (if not in introduction)

\IGSEQ is a precise and quantitative method which is highly sensitive to biases and errors arising during both library preparation (especially PCR) and sequencing. In order to minimise the impact of these errors on the results, unique molecular identifiers (UMIs) are widely used in \igseq library-preparation protocols to uniquely label each input molecule, enabling sequences in the dataset arising from the same input molecule to be identified, grouped and collapsed into a single consensus sequence. This process effectively corrects for biases in apparent abundance arising from differential amplification during library-preparation or differential clustering during sequencing, and allows identical sequences descended from the same input molecule to be distinguished from those descended from different input molecules with the same sequence. In addition, as long as enough reads arising from a given input sequence are present, the taking of a consensus sequence from each UMI group effectively corrects for sequence errors arising during library-preparation and sequencing, enabling the original input sequence to be reconstructed much more accurately (\Cref{fig:umi-consensus-schema}).
% TODO: Cite Vollmers, Turchaninova

In the \igseq protocol established here for the turquoise killifish, the addition of UMIs is achieved through the use of a procedure adapted from \parencite{turchaninova2016igprep}. This procedure takes advantage of the intrinsic terminal transferase activity exhibited by reverse-transcriptase enzymes derived from Moloney Murine Leukemia Virus (MMLV), which includes most commercially-available reverse-transcriptase enzymes. Upon reaching the end of an RNA template, such viruses add a variable number of untemplated deoxyribonucleotides to the 3'-end of the new cDNA molecule, with a strong bias for cytidine residues. If a template-switch adapter (TSA) oligonucleotide ending in riboguanosines is added to the reaction mixture, it will pair with these untemplated terminal cytidines to form a new priming site for the reverse transcriptase. The enzyme will then re-attach to this new priming site and process to the end of the paired oligo, adding the sequence of the TSA to the 3'-end of the cDNA. This procedure, known as template switching (\Cref{fig:template-switch-schema}), enables semi-arbitrary sequences to be prepended to the reverse-transcribed mRNA sequence, including a primer sequence and (in this case) a UMI. % TODO: Citations for MMLV terminal transferase activity, TSA length, template switching; cite Turchaninova at end

\begin{figure}
\centering
\includegraphics[width=0.7\textwidth]{_Figures/png_edited/umi-consensus-schema}
\caption{\textbf{Correcting errors and biases with unique molecular identifiers (UMIs):} In this simplified schematic, three template RNA molecules representing two distinct sequences (red and blue rectangles) are tagged with UMIs (coloured left-hand ends) prior to PCR-based library preparation and sequencing. As a result of differential amplification bias between the sequences, the less-abundant input sequence gives rise to a larger number of sequencing reads; however, by aligning and collapsing matching UMI groups into consensus reads, the original proportions of the sample can be reconstructed. Consensus-read generation also enables the correction of various PCR and sequencing errors (thin red/green bars), the majority of which are present in only a minority of sequence copies within a UMI group; only the dark-blue UMI group, for which only a single sequencing read was obtained, cannot be effectively corrected in this way.}
\label{fig:umi-consensus-schema}
\end{figure}

\begin{figure}
\centering
\includegraphics[width=0.7\textwidth]{_Figures/png_edited/template-switch-labelled}
\vspace{0.5em}
\caption{\textbf{Addition of a known 5'-sequence with template switching:} In this simplified schematic, an MMLV-derived reverse-transcriptase enzyme (blue oval) binds a primed RNA template and processes to the 5-prime end (A), where it deposits additional untemplated cytidine residues at the 3'-end of the CDNA (red bar) using its terminal-transferase activity (B). A template-switch adaptor (TSA) with complementary terminal guanosine residues (blue bar) pairs with these terminal cytidines (C), creating a new double-stranded priming site to which the reverse-transcriptase enzyme can bind (D). Processing of the enzyme to the end of the TSA sequence produces a cDNA molecule with an additional, known 3' sequence (E).}
\label{fig:template-switch-schema}
\end{figure}

In the library-preparation protocol used for \Igseq in the turquoise killifish, therefore, reverse-transcription is performed on total RNA using a gene-specific primer (GSP) homologous to the constant-region sequence of the isotype of interest and an MMLV-derived reverse-transcriptase enzyme optimised for its terminal-transferase activity (\Cref{sec:methods_molec_igseq_rt}). A template-switch adaptor containing a constant primer sequence and random UMI sequence (\Cref{fig:igseq-tsa-sequence}) is included in the reaction mixture and added to the cDNA sequence via template switching. In addition to enabling UMI-based clustering and correction as described above, this approach has the additional advantage of bypassing the variable region of the \igh{} transcript and thus avoiding the use of multiplexed V-segment primers, which may introduce additional biases through differential binding affinity between V-segments; in addition, the template-switching approach is more sensitive than the multiplexed-primers approach to low concentrations of \igh{} transcripts in a sample, making it more suitable for \Igseq on samples which have not been enriched for B-cells. % TODO: Citation for this last claim?

\begin{figure}
\begin{center}
\LARGE
\textcolor{Fuchsia}{AAGCAGUGGTAUCAACGCAGAG}U\textcolor{ForestGreen}{NNNN}--\\--U\textcolor{ForestGreen}{NNNN}U\textcolor{ForestGreen}{NNNN}UCTT\textcolor{BurntOrange}{rGrGrGrG}
\end{center}
\caption{Annotated sequence of the SmartNNNa barcoded template-switch adapter (TSA) used in template-switch reverse-transcription for \igseq library preparation \parencite{turchaninova2016igprep}. The 5'-terminal purple characters represent a constant sequence used for primer-binding in downstream PCR steps, while the green N characters represent the random nucleotides used in the unique molecular identifier (UMI), each of which could take any value from A, C, G or T. The U residues represent deoxyuridine, which is specifically digested after reverse-transcription to remove residual TSA oligos from the reaction mixture (\Cref{sec:methods_molec_igseq_rt}). The orange, 3'-termial rG characters indicate riboguanosine residues, which pair with terminal-transferase-added cytidine residues during template switching.}
\label{fig:igseq-tsa-sequence}
\end{figure}

\begin{figure}
\centering
\caption{\textbf{Summary of \igseq library-preparation protocol}}
\label{fig:igrace-pipeline}
\includegraphics[width=0.7\textwidth]{_Figures/png_edited/igrace-pipeline-narrow}
\vspace{0.5em}
\end{figure}

Following reverse-transcription, the reaction mixture is treated with uracil-DNA glycosylase (UDG) to specifically remove residual TSAs through specific digestion of deoxyuridine residues (which are absent in the template sequence). The library then undergoes three successive rounds of PCR amplification (\Cref{fig:igrace-pipeline}; see\Cref{sec:methods_molec_igseq_pcr} for more details), which respectively serve to amplify the reverse-transcribed cDNA sequence; amplify further while adding partial Illumina TruSeq adaptor sequences; and add complete adaptor sequences including library-specific P1 and P2 index barcode sequences. In the first two cases, the PCR primer sequences are homologous to the invariant part of the TSA sequence and the \cm{1} exon, respectively; as all known forms of \igh{} in turquoise killifish (\igh{M-TM}, \igh{M-S} and \igh{D-TM}) share this exon, the isotype- and isoform-specificity of the library prep can be altered simply by changing the position of the reverse-transcription GSP, with all other primer sequences left unchanged. In all experiments included in this chapter, a GSP on the \cm{2} exon was used, resulting in a library including \igh{M-TM} and \igh{M-S} sequences but excluding \igh{D}. % TODO: Cite Vollmers for PCR-based adaptor addition

Libraries to be sequenced together are then pooled in equimolar ratio and the pooled sample undergoes size-selection to remove residual primer-dimers and other unwanted sequences. The complete library-preparation protocol reliably produces a single peak in the range of 650-\bp{680}, consistent with the known lengths of \vh, \dh and \jh gene segments in the \Nfu locus. % TODO: Figure for this, comparing expected and observed peak length
Following size-selection and quality control, pooled libraries are sequenced on an Illumina MiSeq sequencing machine with 2 × \bp{300} reads (\Cref{sec:methods_molec_igseq_seq}); this longer read length is necessary to completely cover the variable region. To avoid problems associated with the low sequence complexity of single-amplicon sequencing libraries, a large proportion of PhiX spike-in (30\%) was used to increase the sequence complexity of the libraries.

\section{Killifish husbandry and sample preparation methods}
\label{sec:methods_husbandry}

Male turquoise killifish (\nfu, GRZ-AD strain) from a single hatching were raised under standard husbandry conditions \parencite{dodzian2018husbandry} and housed from four weeks post-hatching in individual \lt{2.8} tanks connected to a water recirculation system. Fish received \hr{12} of light per day on a regular light/dark cycle, and were fed blood
worm larvae and brine shrimp nauplii twice a day during the week and once a day during the weekend \parencite{dodzian2018husbandry,smith2017microbiota}.

Sacrificed fish (\Cref{tab:igseq-cohorts-fish}) were killed by anaesthetisation in \gl{1.5} Tricaine solution in room-temperature tank water \parencite{carter2011tricaine}, then flash-frozen in liquid nitrogen and ground to a homogenous powder with a pestle in a liquid-nitrogen-filled mortar. The powder was mixed thoroughly and stored at \degC{-80} prior to RNA isolation.
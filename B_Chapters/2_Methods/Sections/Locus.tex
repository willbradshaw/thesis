\subsubsection{BAC isolation and sequencing}
\label{sec:bac-methods-isol}
% TODO: Check the volumes, speeds etc against old lab book

The identified BAC clones were provided by the FLI institute in Jena as plate or stab cultures of transformed \textit{E. coli}, which were replated and stored at \degC{4} when not in use. Prior to isolation, the clones of interest were cultured overnight in [volume]ml LB medium to produce a large liquid culture. The cultures were then transferred to \ml{50} conical tubes and centrifuged (\mins{5}, \degC{4}, \g{12000}) to pellet the cells. After pelleting, the supernatant was carefully discarded and the cells were resuspended in \ml{10} buffer P1.

After resuspension, the cultures underwent alkaline lysis to release the BAC DNA and precipitate genomic DNA and cellular debris. \ml{10} lysis buffer P2 was added to each tube, which was then mixed gently but thoroughly by inversion and incubated at room temperature for \mins{5}. \ml{10} ice-chilled neutralisation buffer P3 was added and each tube was mixed gently but thoroughly by inversion and incubated on ice for \mins{15}. The tubes were then centrifuged (\mins{20}, \degC{4}, \g{12000}) to pellet cellular debris and the supernatant transferred to new conical tubes. This process was repeated at least two more times, until no more debris was visible in any tube; this repeated pelleting was necessary to minimise contamination in each sample, as the normal column- or paper-based filtering steps used during alkaline lysis protocols result in the loss of the BAC DNA.

Following alkaline lysis, the BAC (and residual genomic) DNA in each sample underwent isopropanol precipitation: 0.6 volumes of room-temperature isopropanol was added to the clean supernatant in each tube, followed by 0.1 volumes of \mol{3} sodium acetate solution. Each tube was mixed well by inversion, incubated for \mins{10} at room temperature, then centrifuged (\mins{30}, \degC{4}, \g{12000}) to pellet the DNA. The supernatant was discarded and the resulting DNA smear was ``resuspended" in \ml{1} \pc{100} ethanol and transferred to a \ml{1.5} tube, which was re-centrifuged (\mins{5}, \degC{4}, \g{20000}) to obtain a concentrated pellet.\dots % TODO: Was this actually what happened

Finally, \dots % This last part differs between BAC groups, so I should check the lab-book for each iteration

%\item Add \x{0.6} room-temperature isopropanol to the clean supernatant, followed my \x{0.1} sodium acetate. Mix well by inversion.
%\item \Incubate{10}{RT},  then \centrifuge{30}{4}{15000}. Promptly but carefully discard the supernatant.
%\item ``Resuspend" the pellet in \ml{1} \pc{100} ethanol and transfer to a \ml{1.5} tube, then centrifuge ($\geq$\mins{5}, \degC{4}, top speed). \alert{hint}
%\item Carefully pour or pipette off the supernatant, then air-dry upside-down for 5-\mins{10}. \alerts{warning}{2}~~\alert{hint}
%\item Resuspend in 30-\ul{50} of preferred buffer. \alert{pause}
%\item Assay BAC isolate yield with the \textbf{Qubit} dsDNA BR assay. Dilute samples to a final volume of c. 50-\ngul{150} and re-quantify. \alert{hint}
%\item Assass BAC isolate quality by running \ul{10} of diluted isolate on a \pc{0.5} \textbf{agarose gel} for 1-\hr{2}, and/or by amplifying the backbone or other query sequence using \textbf{Kapa PCR}. % add figure for QC gel appearance
%\item Proceed to \textbf{shearase-based Illumina library preparation}.
%\end{protocol}
%\end{main}

The resuspended BAC isolates were sent to the Cologne Center for Genomics, where they underwent ... library preparation and were sequenced on an Illumina MiSeq sequencing machine ([read length etc here], 2x300bp reads).
